\documentclass[12pt, a4paper]{article}
\usepackage[utf8]{inputenc}
\usepackage[french]{babel}
\usepackage{geometry}
\usepackage{graphicx}
\usepackage{listings}
\usepackage{xcolor}
\usepackage{hyperref}
\usepackage{amsmath}

\geometry{margin=1in}

% Configuration des listings pour le code
\lstset{
    language=Java,
    basicstyle=\ttfamily\small,
    breaklines=true,
    keywordstyle=\color{blue},
    commentstyle=\color{gray},
    stringstyle=\color{red},
    backgroundcolor=\color{lightgray!10},
    frame=single,
    numbers=left,
    numberstyle=\tiny
}

\title{Rapport TP3 - Code de Hamming}
\author{Nathan ADOHO}
\date{\today}

\begin{document}

\maketitle

\tableofcontents
\newpage

\section{Introduction}
% À compléter
L'oblectif de ce TP était de :
- comprendre et d'implémenter le code de Hamming pour la detection d'erreurs;
- déterminer lorsque c'est possible le code à ajoindre à un message pour en faire un mot de Hamming et ce dernier.

\section{Objectifs}
% À compléter
\begin{itemize}
    \item Objectif 1
    \item Objectif 2
    \item Objectif 3
\end{itemize}

\section{Concepts Théoriques}
% À compléter

\subsection{Code de Hamming}
% À compléter
Décrire le fonctionnement du code de Hamming.

\subsection{Détection et Correction d'Erreurs}
% À compléter
Expliquer comment les erreurs sont détectées et corrigées.

\section{Implémentation}

\subsection{Architecture}
% À compléter
Décrivez l'architecture générale de votre implémentation.

\subsection{Classes Principales}
% À compléter

\subsubsection{Classe Hamming}
\lstinputlisting[caption=Hamming.java]{../Hamming.java}

\subsubsection{Classe Fenetre}
\lstinputlisting[caption=Fenetre.java]{../Fenetre.java}

\section{Résultats et Tests}
% À compléter
Présentez les résultats de vos tests et des démonstrations.

\subsection{Cas de Test}
% À compléter

\subsection{Résultats Obtenus}
% À compléter

\section{Conclusion}
% À compléter
Résumez les apprentissages et difficultés rencontrées.

\section{Annexes}
% À compléter selon les besoins

\end{document}
